\hypertarget{background_backgroundoverview}{}\section{Overview}\label{background_backgroundoverview}
This library provides routines for regression and forward model problems using trans-\/dimensional Markov chain Monte Carlo methods. Rather than producing a single fit for a dataset, these methods produce an ensemble of results that give a sampled distribution of the potential true fit. The original paper describing this approach is by Green {\bfseries [green1995]}, and this software is primarily based on the work of Bodin and Sambridge {\bfseries [sambridge2006A]} {\bfseries [bodin\+Thesis]} {\bfseries [bodin2012A]} .\hypertarget{background_backgroundprior}{}\section{Prior}\label{background_backgroundprior}
The general approach taken in all the routines within this library is to use uniform priors on all values.

For partition locations, we use a symmetric Dirichlet prior which is described in Steininger {\bfseries [steininger2013]} and references therein.\hypertarget{background_backgroundmodel}{}\section{Model}\label{background_backgroundmodel}
For 1D applications we use zeroth order, natural, and polynomial models within each partition.

Zeroth order 1D models are described in {\bfseries [bodin\+Thesis]} {\bfseries [bodin2012A]}.

Natural uses jointed line segments between partition boundaries to create a C0 continuous curve over the domain of the entire model. It is described in Hopcroft {\bfseries [hopcroft2007A]} .

For 1D regression problems only, we also use the data within each partition to inform the selection of a suitable polynomial order (up to a limit imposed by the user), ie trans-\/dimensional within each partition. The mechanism for choosing this order is described in Sambridge {\bfseries [sambridge2006A]} and the order prior is set to uniform over a region determined from the data mean and standard deviation within the partition.

For 2D applications at present only zeroth order partitions are used as described in {\bfseries [bodin2012B]} and {\bfseries [bodin2012C]}.\hypertarget{background_backgroundproposal}{}\section{Proposal}\label{background_backgroundproposal}
All proposals used in this library use pertubations sampled from a Gaussian random variable. The Standard deviation is generally set as a user parameter, the only exception to this is in the case of 1D Regression routines where the standard deviation is obtained directly from the data.\hypertarget{background_backgroundhierarchical}{}\section{Hierarchical Parameter Estimation}\label{background_backgroundhierarchical}
Hierarchical parameters are parameters used to quantify the noise in the data. In the simple regression case, we allow the use of a single hierarchical scaling factor called lambda that represents a multiplier of the estimated data error.

In the general forward model case, we allow a general mechanism for hierarchical parameter estimation requiring the forward model to calculate the log of the determinant of the data covariance matrix.

For an example of more complex hierarchical parameter estimation, see Bodin {\bfseries [bodin2012A]} . 